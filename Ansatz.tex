\documentclass{article}
\usepackage[utf8]{inputenc}
\usepackage{enumitem}

\title{Ansatz für Goldpreisvorhersage mit Technischer Analyse}
\author{}
\date{}

\begin{document}

\maketitle


\begin{itemize}[noitemsep]
    \item Datenbeschaffung: Historische Goldpreisdaten von einer verlässlichen Quelle oder API beschaffen.
    \item Datenbereinigung: Fehlende Werte behandeln, Duplikate entfernen und eventuell vorhandene Anomalien oder Ausreißer untersuchen.
    \item Berechnung technischer Indikatoren: Gleitende Durchschnitte (SMA, EMA), Relative Strength Index (RSI), Moving Average Convergence Divergence (MACD), Bollinger-Bänder berechnen.
    \item Datenanalyse: Muster oder Zusammenhänge zwischen Indikatoren und Goldpreisbewegungen erkennen.
    \item Zeitreihenanalyse: Autoregressive Integrated Moving Average (ARIMA) oder Exponential Smoothing State Space Model (ETS) verwenden, um ein Modell für die Goldpreisprognose zu erstellen.
    \item Regressionsanalyse: Linearer Regression oder Lasso-Regression verwenden, um die Beziehung zwischen technischen Indikatoren und Goldpreis zu modellieren.
    \item Modellvalidierung und Auswahl: Leistung der Zeitreihen- und Regressionsmodelle vergleichen, um das beste Modell für die Goldpreisprognose auszuwählen.
    \item Vorhersage und Evaluation: Ausgewähltes Modell verwenden, um zukünftige Goldpreise vorherzusagen. Leistung des Modells regelmäßig evaluieren und bei Bedarf anpassen.
\end{itemize}

\end{document}
